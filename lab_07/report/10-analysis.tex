\chapter{Аналитическая часть}
В данном разделе будут описаны словарь как структура данных и алгоритм поиска в словаре.

\section{Структура данных словарь}

Словарь --- абстрактный тип данных, позволяющий хранить пары вида «(ключ, значение)» и поддерживающий операции добавления пары, а также поиска и удаления пары по ключу~\cite{dict}:
\begin{enumerate}[label=\arabic*)]
	\item \textit{insert(key, val)};
	\item \textit{find(key)};
	\item \textit{remove(key)}.
\end{enumerate}

В паре \textit{(key, val)}: \textit{val} называется значением, ассоциированным с ключом \textit{key}. Здесь \textit{key} --- это ключ, a \textit{val} --- значение. Семантика и названия вышеупомянутых операций в разных реализациях ассоциативного массива могут отличаться.

Операция поиска \textit{find(key)} возвращает значение, ассоциированное с заданным ключом, или некоторый специальный объект, означающий, что значения, ассоциированного с заданным ключом, нет. Две другие операции ничего не возвращают (только информацию об успешности выполнения операции).

\section{Алгоритм приска в словаре}
В качестве алгоритма поиска в словаре был выбран алгоритм полного перебора.
Алгоритм полного перебора --- это алгоритм разрешения математических задач, который можно отнести к классу способов нахождения решения рассмотрением всех возможных вариантов~\cite{all_alg}.
В случае реализации алгоритма в рамках данной работы будут последовательно перебираться ключи словаря до тех пор, пока не будет найден нужный.

Трудоёмкость алгоритма зависит от того, присутствует ли искомый ключ в словаре, и, если присутствует --- насколько он далеко от начала массива ключей.

Пусть в начале свое работы производит $n_{0}$ операций, а при сравнении $n_{1}$ операций.
Лучший случай происходит, когда алгоритм находит нужный элемент при первом сравнении, будет произведено $n_0 + n_1$ операций, худший случай, когда алгоритм находит нужный элемент, перебрав все элементы или если данного ключа нет в массиве, то будет произведено $n_0 + N \cdot n_1$ операций, где $N$ --- общее число ключей в словаре. Трудоёмкость в среднем может быть рассчитана как математическое ожидание по формуле (\ref{for:brute}), где $\Omega$ --- множество всех возможных случаев.

\begin{equation}
	\label{for:brute}
	\begin{aligned}
		\sum\limits_{i \in \Omega} p_i \cdot f_i = n_0 + n_1 \cdot \left(1 + \frac{N}{2} - \frac{1}{N + 1}\right)
	\end{aligned}
\end{equation}

\section*{Вывод}
В данном разделе были описаны словарь как структура данных и алгоритм поиска в словаре.
