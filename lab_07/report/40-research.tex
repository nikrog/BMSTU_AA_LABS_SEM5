\chapter{Исследовательская часть}
В текущем разделе будут представлены пример работы разработанного программного обеспечения, постановка эксперимента и описание полученных результатов.

\section{Пример работы программного обеспечения}

На рисунках \ref{img:req1.PNG} -- \ref{img:req3.PNG} представлены результаты работы программы, на вход программе подается вопрос, если соответствующий терм и объект найдены в вопросе, то выводится информация о футболистах (трансферная стоимость, фамилия, страна рождения, игровая позиция) с заданным диапазоном трансферной стоимости иначе выводится ошибка.

\imgw{\textwidth}{req1.PNG}{Пример работы программы}
\imgw{\textwidth}{req2.PNG}{Пример работы программы}
\imgw{\textwidth}{req3.PNG}{Пример работы программы}


\section{Формализация объекта и его признака}
\label{formal}
Согласно индивидуальному варианту, формализуем объект <<футболист>> следующим образом: определим набор данных и признак объекта, на основании которого составим набор термов.

Набор данных:
\begin{enumerate}[label=\arabic*)]
	\item фамилия футболиста --- строка;
	\item страна рождения --- строка;
	\item игровая позиция --- строка.
\end{enumerate}
Признаком, по которому будет производиться поиск объектов, будет \textit{трансферная стоимость} в миллионах евро --- вещественное число.

Определим следующие термы, соответствующие признаку <<трансферная стоимость>>:
\begin{enumerate}[label=\arabic*)]
	\item <<Очень низкая>>;
	\item <<Низкая>>;
	\item <<Средняя>>;
	\item <<Высокая>>;
	\item <<Очень высокая>>.
\end{enumerate}

Также введём для данной задачи интервал оцениваемой величины (трансферной стоимости) $P$:
\begin{equation}
	\label{eq:h}
	P \in (0, 200)
\end{equation}

\section{Анкетирование респондентов}

Было проведено анкетирование следующих респондентов:
\begin{enumerate}[label=\arabic*)]
	\item Косарев Алексей, группа ИУ7-51Б --- Респондент 1;
	\item Котляров Никита, группа ИУ7-51Б --- Респондент 2;
	\item Корниенко Клим, группа ИУ7-51Б --- Респондент 3;
	\item Кормановский Михаил, группа ИУ7-51Б --- Респондент 4;
	\item Никулина Анна, группа ИУ7-51Б --- Респондент 5;
	\item Бурлаков Илья, группа ИУ7-51Б --- Респондент 6;
        \item Кузьмин Серафим, группа ИУ7-51Б --- Респондент 7.
\end{enumerate}

Респонденты, выступающие в качестве экспертов, для каждого из приведённых выше термов указали соответствующий промежуток из введенного для поставленной задачи интервала оцениваемой величины.

Результаты анкетирования перечисленных респондентов представлены в таблице~\ref{table:anket}. В данной таблице Р. --- сокращение от <<Респондент>>, Т. 1 -- 5 --- термы, соответствующие обозначенным в п. \ref{formal} термам.

\captionsetup{singlelinecheck = false, justification=raggedright}
\begin{center}
	\captionsetup{justification=raggedright,singlelinecheck=off}
	\begin{table}[ht]
		\centering
		\caption{Тестирование реализации алгоритма поиска в словаре полным перебором}
		\label{table:anket}
		\begin{tabular}{|c|c|c|c|c|c|}
			\hline
			   & Т. 1  	  & Т. 2 &  Т. 3 & Т. 4 & Т. 5\\ 
			\hline
			Р. 1 & (0, 10) & [10, 60) & [60, 100)   & [100, 160) & [160, 200)\\ \hline
			Р. 2 & (0, 5)  & [5, 50) & [50, 120)  & [120, 160) & [160, 200)\\ \hline
			Р. 3 & (0, 8) & [8, 40) & [40, 110) & [110, 150) & [150, 200)\\ \hline
			Р. 4 & (0, 15) & [15, 70) & [70, 130) & [130, 185) & [185, 200)\\ \hline
			Р. 5 & (0, 10) & [10, 40) & [40, 100) & [100, 170) & [170, 200)\\ \hline
                Р. 6 & (0, 1) & [1, 15) & [15, 60) & [60, 100) & [100, 200)\\ \hline
                Р. 7 & (0, 10) & [10, 70) & [70, 100) & [100, 150) & [150, 200)\\ \hline
		\end{tabular}
	\end{table}
\end{center}
\captionsetup{singlelinecheck = false, justification=centering}
\section{Построение функции принадлежности термам}

Для каждого целого значения, кратного 10 из $P$ для каждого терма из перечисленных будет найдено количество респондентов, согласно которым данное значение из $P$ удоволетворяет сопоставляемому терму.
Это значение, поделенное на количество респондентов --- значение функции $\mu$ для терма в точке.
Графики функций принадлежности числовых значений трансферной стоимости термам, представлен на рисунке \ref{img:term_graph2.png}.


\imgw{\textwidth}{term_graph2.png}{Графики функций принадлежности числовых значений переменной термам, описывающим группы значений лингвистической переменной}

В соответствии с полученным графиком будем считать трансферную стоимость футболистов:
\begin{enumerate}[label=\arabic*)]
	\item очень низкой, если ее значение лежит в промежутке $(0; 10]$ миллионов евро;
	\item низкой, если ее значение лежит в промежутке $(10; 50]$ миллионов евро;
	\item средней, если ее значение лежит в промежутке $(50; 110]$ миллионов евро;
	\item высокой, если ее значение лежит в промежутке $(110; 150]$ миллионов евро;
	\item очень высокой, если ее значение лежит в промежутке $(150; 200)$ миллионов евро.
\end{enumerate}

\section*{Вывод}
В данном разделе были представлены примеры работы разработанного программного обеспечения, постановка эксперимента и и описание его результатов.
