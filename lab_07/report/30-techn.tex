\chapter{Технологическая часть}
В данном разделе будут представлены требования к программному обеспечению, средства реализации, листинги кода и тесты.

\section{Требования к программному обеспечению}

Вход: строка (вопрос), на основании которой будет производиться поиск.

Выход: результат поиска в словаре (футболисты с заданным диапазоном трансферной стоимости).

\section{Выбор средств реализации}

В качестве языка программирования для реализации данной лабораторной работы был выбран язык программирования Python  \cite{PythonBook}. В данном языке программирования существует встроенный тип данных словарь (dict), также данный язык имеет необходимые библиотеки для построения графиков.

В качестве среды разработки был выбран PyCharm Professional \cite{pycharm}. Данная среда разработки является кросс-платформенной, предоставляет функциональный отладчик, средства для рефакторинга кода и возможность установки необходимых библиотек при необходимости.

\section{Реализация алгоритма}
В листингах \ref{full} представлена реализация алгоритма поиска в словаре полным перебором.

\captionsetup{singlelinecheck = false, justification=raggedright}
\begin{lstlisting}[caption=Реализация алгоритма поискав словаре полным перебором, 
    label={full}]
		def full_search(self, key):
			val = NOT_FOUND
			keys = list(self.data.keys())
			for k in keys:
				    if key == k:
					   val = self.data[k]
                          break
			return val
\end{lstlisting}

\section{Тестирование}

В таблице \ref{table:test} приведены функциональные тесты для алгоритма поиска в словаре полным перебором. Все тесты были пройдены успешно.

\begin{center}
	\captionsetup{justification=raggedright,singlelinecheck=off}
	\begin{table}[ht]
		\centering
		\caption{Тестирование реализации алгоритма поиска в словаре полным перебором}
		\label{table:test}
		\begin{tabular}{ |c|c|c|}
			\hline
			Входные данные    & Ожид. результат   	  & Фактич. результат    \\ 
			\hline
			очень низкая			  & (0, 10]   & (0, 10] \\ \hline
			высокая 			  & [111, 150]    & [111, 150] \\ \hline
			средняя 		  & [51, 110] & [51, 110] \\ \hline
			хорошая & Не найдено & Не найдено \\ \hline
			111 & Не найдено & Не найдено \\ \hline
		\end{tabular}
	\end{table}
\end{center}
\captionsetup{singlelinecheck = false, justification=centering}

\section*{Вывод}
В данном разделе были представлены требования к программному обеспечению и средства реализации, реализован и протестирован алгоритм поиска в словаре полным перебором.

