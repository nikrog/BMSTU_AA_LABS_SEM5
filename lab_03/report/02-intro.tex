\chapter*{Введение}
\addcontentsline{toc}{chapter}{Введение}
\textbf{Цель лабораторной работы} - получение навыков оценки трудоемкости алгоритмов на материале трех алгоритмов сортировки: гномья, поразрядная и выбором.

\textbf{Алгоритм сортировки} — это алгоритм для упорядочивания элементов определенным образом в заданной последовательности. 
Одной из главных целей сортировки является упрощение задачи поиска элемента в отсортированной последовательности.

Сортировка является одним из важнейших классов алгоритмов обработки данных и осуществляется большим количеством способов ~\cite{sorts_book}.

Алгоритмы сортировки имеют большое практическое применение и часто встречаются там, где нужно обрабатывать и храненить большие объемы информации.
Часто обрабатывать данные проще, их заранее упорядочить.
Упорядоченные данные содержатся, например, в библиотеках, словарях, различных архивах.

В настоящее время проблема эффективной сортировки остается актуальной из-за постоянно растущих объемов данных.

Для достижения поставленной цели требуется решить следующие задачи:
\begin{enumerate}[label={\arabic*)}]
    \item изучение трех алгоритмов сортировки: гномья сортировка, поразрядная сортировка, сортировка выбором;
    \item разработка трех алгоритмов сортировки: гномья сортировка, поразрядная сортировка, сортировка выбором;
    \item реализация трех алгоритмов сортировки: гномья сортировка, поразрядная сортировка, сортировка выбором;
    \item выполнение замеров процессорного времени работы реализаций алгоритмов сортировки: гномья сортировка, поразрядная сортировка, сортировка выбором;
	\item сравнительный анализ трудоемкости реализаций разработанных алгоритмов сортировки на основе теоретических расчетов;
	\item сравнительный анализ процессорного времени работы реализаций разработанных алгоритмов сортировки на основе экспериментальных данных.
\end{enumerate}
