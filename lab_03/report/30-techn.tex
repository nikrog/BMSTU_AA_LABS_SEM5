\chapter{Технологическая часть}

В данном разделе будут представлены требования к программному обеспечению, средства реализации, листинги кода и тесты.

\section{Требования к программному обеспечению}
Вход: массив целых чисел, количество чисел в массиве;

Выход: тот же массив, но уже отсортированный.

Результат сортировки массива должен быть выведен для каждого из реализованных алгоритмов сортировки (гномья сортировка, поразрядная сортировка, сортировка выбором).

\section{Выбор средств реализации}

В качестве языка программирования для реализации данной лабораторной работы был выбран язык программирования Python  \cite{PythonBook}. Данный язык ускоряет процесс разработки и удобен в использовании.

Процессорное время реализованных алгоритмов было замерено с помощью функции process\_time() из библиотеки time \cite{process_time_text}.

В качестве среды разработки был выбран PyCharm Professional \cite{pycharm}. Данная среда разработки является кросс-платформенной, предоставляет функциональный отладчик, средства для рефакторинга кода и возможность быстрой установки необходимых библиотек при необходимости.

\section{Реализация алгоритмов}

В листингах \ref{gnome_sort} - \ref{selection_sort} представлены реализации различных алгоритмов сортировки: гномьей сортировки, поразрядной сортировки, сортировки выбором.

\begin{lstlisting}[caption=Функция алгоритма гномьей сортировки, 
    label={gnome_sort}]
def gnome_sort(arr):
    i = 1
    n = len(arr)
    while i < n:
        if arr[i - 1] > arr[i]:
            arr[i - 1], arr[i] = arr[i], arr[i - 1]
            if i > 1:
                i -= 1
        else:
            i += 1
    return arr
\end{lstlisting}


\begin{lstlisting}[caption=Функция алгоритма поразрядной сортировки (классическая реализация для сортировки целых неотрицательных чисел),
    label={radix_sort}]
# counting quantity of digits in number
def count_digits(num):
    c = 0
    while num > 0:
        num /= 10
        c += 1
    return c


# counting quantity of digits in maximum number in array
def num_digits(arr):
    max_num = arr[0]
    n = len(arr)
    for i in range(n):
        if max_num < arr[i]:
            max_num = arr[i]
    return count_digits(max_num)


# main function
def radix_sort(arr):
    m_dig = num_digits(arr)
    for d in range(0, m_dig):
        # 10, because of 10 possible digits (0 - 9)
        tmp = [[] for i in range(10)]
        for i in range(len(arr)):
            num = (arr[i] // (10 ** d)) % 10
            tmp[num].append(arr[i])
        arr = reduce(lambda x, y: x + y, tmp)
    return arr

\end{lstlisting}

\begin{lstlisting}[caption=Функция алгоритма поразрядной сортировки (модифицированная реализация для сортировки целых чисел),
	label={radix_sort2}]
# counting quantity of digits in number
def count_digits(num):
    c = 0
    num = abs(num)
    while num > 0:
        num //= 10
        c += 1
    return c


# counting quantity of digits in maximum number in array
def num_digits(arr):
    max_num = abs(arr[0])
    n = len(arr)
    for i in range(n):
        if max_num < abs(arr[i]):
            max_num = abs(arr[i])
    return count_digits(max_num)


# main function
def radix_sort(arr):
    m_dig = num_digits(arr)
    for d in range(0, m_dig):
        # 10, because of 10 possible digits (0...9),
        # but for negative numbers add digits (-9...-1)
        tmp = [[] for i in range(19)]
        for i in range(len(arr)):
            if arr[i] < 0:
                num = -((abs(arr[i]) // (10 ** d)) % 10)
            else:
                num = (arr[i] // (10 ** d)) % 10
            tmp[9 + num].append(arr[i])
        arr = reduce(lambda x, y: x + y, tmp)
    return arr
\end{lstlisting}

\begin{lstlisting}[caption=Функция алгоритма сортировки выбором,
	label={selection_sort}]
def selection_sort(arr):
    n = len(arr)
    for i in range(n - 1, 0, -1):
        i_max = i
        for j in range(i - 1, -1, -1):
            if arr[j] > arr[i_max]:
                i_max = j
        if i_max != i:
            arr[i], arr[i_max] = arr[i_max], arr[i]
    return arr
\end{lstlisting}

\section{Тестирование}

В таблице \ref{test} приведены функциональные тесты для алгоритмов сортировки (гномья сортировка, поразрядная сортировка, сортировка выбором). Все тесты были пройдены успешно.

\begin{table}[h]
	\begin{center}
		\caption{\label{test} Тесты}
\begin{tabular}{| c | l | l | }
	\hline
	№ & Исходный массив & Ожидаемый результат \\ \hline
	1 & 0 & 0 \\
	\hline
	2 & 5, 5, 5, 5 & 5, 5, 5, 5 \\
	\hline
	3 & 1, 45, 78, 109, 764 & 1, 45, 78, 109, 764 \\
	\hline
	4 & 78, 44, 32, 1 & 1, 32, 44, 78 \\
	\hline
	5 & -90, -356, -4, -1058 & -1058, -356, -90, -4\\
	\hline
	6 & 234, 7, -11, 68, -3, 1005 & -11, -3, 7, 68, 234, 1005\\
	\hline
\end{tabular}
	\end{center}
\end{table}

\section*{Вывод}

В данном разделе были представлены требования к программному обеспечению и средства реализации, реализованы и протестированы алгоритмы сортиовки: гномья сортировка, поразрядная сортировка, сортировка выбором.

