\chapter{Аналитическая часть}
В данном разделе будут представлены описания алгоритмов сортировки: гномья сортировка, поразрядная сортировка, сортировка выбором.

\section{Гномья сортировка}
Алгоритм гномьей сортировки похож на сортировку вставками, но в отличие от последней, перед вставкой на нужное место происходит серия обменов, как в сортировке пузырьком.\cite{sorts_book}

Общие идеи алгоритма:
\begin{itemize}
	\item обход массива ведется слева направо (от начала массива до его конца), аналогично пузырьковой сортировке сравниваются соседние элементы и меняются местами, если левое значение больше правого;
	\item если два соседних элемента пришлось поменять местами, то делается шаг назад на 1 элемент.
\end{itemize}

\section{Поразрядная сортировка}
Алгоритм поразрядной сортировки сильно отличается от других алгоритмов сортировки \cite{radix_sort}.

Во-первых, он совсем не использует сравнений сортируемых элементов.

Во-вторых, ключ, по которому происходит сортировка, необходимо разделить на части, разряды ключа (число делится на цифры, слова делятся на буквы).

До начала сортировки необходимо знать два параметра: k и m, где
\begin{itemize}
    \item k - количество разрядов в самом длинном ключе;
    \item m - разрядность данных, то есть количество возможных значений разряда ключа.
\end{itemize}

Например, при сортировке десятичных чисел m=10, так как десятичная цифра может принимать не более 10 значений (от 0 до 9). Если в самом длинном числе 12 цифр, то k=12.

Эти параметры нельзя изменять в процессе работы алгоритма.

Основная идея алгоритма:

Cравнение производится поразрядно, а именно сначала сравниваются значения одного крайнего разряда, и элементы группируются по результатам этого сравнения, затем сравниваются значения следующего разряда, соседнего, и элементы либо упорядочиваются по результатам сравнения значений этого разряда внутри образованных на предыдущем проходе групп, либо переупорядочиваются в целом, но сохраняя относительный порядок, достигнутый при предыдущей сортировке. Затем аналогично делается для следующего разряда, и так до конца.

%Так как выравнивать сравниваемые записи относительно друг друга можно в разную -- сторону, на практике существуют два варианта этой сортировки: можно выровнять -- записи чисел в сторону менее значащих цифр (least significant digit, LSD) или -- более значащих цифр (most significant digit, MSD).

\section{Сортировка выбором}
Алгоритм сортировки выбором основан на сравнении каждого элемента с каждым, в случае необходимости производится обмен \cite{select_sort}.

Шаги выполнения алгоритма.
\begin{enumerate}
    \item Проходим по массиву в поисках максимального элемента, запоминаем его номер;
    \item Найденный максимум меняем местами с последним элементом (обмен не нужен, если максимальный элемент уже находится на нужной позиции);
    \item Неотсортированная часть массива уменьшилась на один элемент (не включает последний элемент, куда мы переставили найденный максимум);
    \item К неотсортированной части применяем те же действия, то есть находим максимум и ставим его на последнее место в неотсортированной части массива;
	\item Продолжаем таким образом до тех пор, пока неотсортированная часть массива не уменьшится до одного элемента.
\end{enumerate}

Для устойчивости алгоритма необходимо в пункте 2 максимальный элемент непосредственно вставлять в последнюю неотсортированную позицию, не меняя порядок остальных элементов, иначе число обменов может резко возрасти.

\section*{Вывод}

В данном разделе были рассмотрены основные идеи, лежащие в основе рассматриваемых алгоритмов сортировки (гномья сортировка, поразрядная сортировка, сортировка выбором).