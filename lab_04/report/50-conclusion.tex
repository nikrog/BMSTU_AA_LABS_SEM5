\chapter*{Заключение}
\addcontentsline{toc}{chapter}{Заключение}

В результате выполнения лабораторной работы цель достигнута: изучена организация параллельных вычислений на базе алгоритма построения спектра отрезков по Брезенхему.

В ходе выполнения данной работы были решены все задачи:

\begin{itemize}
    \item изучены основы параллельных вычислений (многопоточности);
    \item изучен алгоритм Брезенхема для построения отрезков;
	\item разработаны параллельная и последовательная версии алгоритма построения спектра отрезков по Брезенхему (то есть с использованием многопоточности и без нее);
	\item реализованы параллельная и последовательная версии алгоритма построения спектра отрезков по Брезенхему;
	\item выполнены замеры процессорного времени работы реализаций алгоритмов сортировки (гномья сортировка, поразрядная сортировка, сортировка выбором);
	\item проведен сравнительный анализ времени выполнения реализаций последовательной и параллельной (при различном количестве потоков) версий алгоритма построения спектра отрезков по Брезенхему;	
	\item проведен сравнительный анализ времени выполнения реализаций с использованием многопоточности и без при разной длине отрезка в спектре.
\end{itemize}

В результате лабораторной работы можно сделать вывод, что самой эффективной по времени является многопоточная реализация (с 4 потоками) алгоритма построения спектра отрезков по Брезенхему, так как ноутбук, на котором производились замеры времени, имеет 4 логических ядра.
