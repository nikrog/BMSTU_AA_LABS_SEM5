\chapter{Конструкторская часть}
В данном разделе будут представлены схемы алгоритмов построения спектра отрезков по Брезенхему с распараллеливанием и без него.
 

\section{Алгоритм Брезенхема построения отрезка}

На рисунке \ref{img:alg1v2.png} приведена схема алгоритма Брезенхема построения отрезка.
\\
\\
\\
\\
\\
\\
\\
\\
\\
\\
\\
\\
\\
\\
\\
\\
\\
\\
\\
\\
\\
\\

\img{240mm}{alg1v2.png}{Схема алгоритма Брезенхема построения отрезка}


\FloatBarrier
\section{Алгоритм построения спектра отрезков по Брезенхему (последовательный)}
На рисунке \ref{img:alg2.png} приведена схема алгоритма построения спектра отрезков по Брезенхему без многопоточности (последовательный).
\\
\\
\\
\\
\\
\\
\\
\\
\\
\\
\\
\\
\\
\\
\\
\\
\\
\\
\\
\\
\\
\\
\\
\\
\\
\\
\\
\\

\img{230mm}{alg2.png}{Схема алгоритма построения спектра отрезков по Брезенхему без многопоточности}


\FloatBarrier
\section{Алгоритм построения спектра отрезков по Брезенхему (параллельный)}
На рисунках \ref{img:alg4.png} и \ref{img:alg3.png} приведена схема алгоритма построения спектра отрезков по Брезенхему с многопоточностью (параллельный).
\\
\\
\\
\\
\\
\\
\\
\\
\\
\\
\\
\\
\\
\\
\\
\\
\\
\\
\\
\\
\\
\\
\\
\\
\\
\\
\\
\\

\img{230mm}{alg4.png}{Схема алгоритма построения спектра отрезков по Брезенхему с многопоточностью}
\img{230mm}{alg3.png}{Схема алгоритма построения спектра отрезков по Брезенхему с многопоточностью}

\FloatBarrier

\section*{Вывод}

В данном разделе были разработаны схемы алгоритмов построения спектра отрезков по Брезенхему с распараллеливанием и без него.