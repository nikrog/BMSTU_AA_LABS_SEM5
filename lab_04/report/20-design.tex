\chapter{Конструкторская часть}
В данном разделе будут представлены схемы алгоритмов построения спектра отрезков по Брезенхему с распараллеливанием и без него.
 

\section{Алгоритм Брезенхема построения отрезка}

На рисунке \ref{img:alg1v3.png} приведена схема алгоритма Брезенхема построения отрезка.
\\
\\
\\
\\
\\
\\
\\
\\
\\
\\
\\
\\
\\
\\
\\
\\
\\
\\
\\
\\
\\
\\

\img{240mm}{alg1v3.png}{Схема алгоритма Брезенхема построения отрезка}


\FloatBarrier
\section{Последовательный алгоритм построения спектра отрезков по Брезенхему}
На рисунке \ref{img:alg2v2.png} приведена схема последовательного алгоритма построения спектра отрезков по Брезенхему, то есть без многопоточности.
\\
\\
\\
\\
\\
\\
\\
\\
\\
\\
\\
\\
\\
\\
\\
\\
\\
\\
\\
\\
\\
\\
\\
\\
\\
\\
\\
\\

\img{230mm}{alg2v2.png}{Схема алгоритма построения спектра отрезков по Брезенхему без многопоточности}


\FloatBarrier
\section{Параллельный алгоритм построения спектра отрезков по Брезенхему}
На рисунках \ref{img:alg3v2.png} и \ref{img:alg4v2.png} приведена схема параллельного алгоритма построения спектра отрезков по Брезенхему, то есть с многопоточностью.
\\
\\
\\
\\
\\
\\
\\
\\
\\
\\
\\
\\
\\
\\
\\
\\
\\
\\
\\
\\
\\
\\
\\
\\
\\
\\
\\
\\

\img{230mm}{alg3v2.png}{Схема алгоритма построения спектра отрезков по Брезенхему с многопоточностью}
\img{230mm}{alg4v2.png}{Схема алгоритма построения спектра отрезков по Брезенхему с многопоточностью}

\FloatBarrier

\section*{Вывод}

В данном разделе были разработаны схемы алгоритмов построения спектра отрезков по Брезенхему с распараллеливанием и без него.