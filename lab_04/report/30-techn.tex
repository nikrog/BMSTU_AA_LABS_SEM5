\chapter{Технологическая часть}

В данном разделе будут представлены требования к программному обеспечению, средства реализации, листинги кода и тесты.

\section{Требования к программному обеспечению}
Пользователю предоставляется графический интерфейс для ввода данных.

Вход: длина отрезка (целое число), режим построения отрезка (с распараллеливанием или без него), количество потоков (в случае выбора режима построения отрезка с распараллеливанием), действие (построение спектра отрезков, очистка экрана, замер процессорного времени работы реализаций алгоритмов построения спектра отрезков);

Выход: спектр отрезков на эране/результаты замеров процессорного времени в консоли.

\section{Выбор средств реализации}

В качестве языка программирования для реализации данной лабораторной работы был выбран язык программирования С++ \cite{cpp}. Данный язык программирования позволяет замерить время выполнения работы реализации алгоритма, а также организовать многопоточность для параллельных вычислений.

Время выполнения реализаций алгоритма построения спектра отрезков было замерено с помощью функции \textit{std::chrono::system\_clock::now(...)} из библиотеки $chrono$ \cite{chrono}.

В качестве среды разработки был выбран QtCreator \cite{qt}. 

Графики были построены с использованием языка программирования Python \cite{PythonBook}.

\section{Реализация алгоритмов}

В листинге \ref{bres_alg} представлена реализация алгоритма Брезенехема для построения отрезка, а в листингах \ref{no_par}, \ref{par} --- реализация алгоритмов построения спектра отрезков по Брезенехему с многопоточностью и без нее соответственно.

\begin{lstlisting}[caption=Функция алгоритма Брезенехема для построения отрезка, 
    label={bres_alg}]
void bresenham_float(const request_t &r, const point_t &p1, const point_t &p2)
{
    int dx = abs(p2.x - p1.x);
    int dy = abs(p2.y - p1.y);

    int sx = sign(p2.x - p1.x);
    int sy = sign(p2.y - p1.y);

    int change_fl = 0;

    if (dy > dx)
    {
        swap(dx, dy);
        change_fl = 1;
    }

    double m = (double)dy / (double)dx;
    double e = m - 0.5;

    int x = p1.x;
    int y = p1.y;

    std::mutex mut;

    for (int i = 0; i < dx; i++)
    {
        if (r.is_draw)
        {
            mut.lock();
            draw_pixel(r, x, y);
            mut.unlock();
        }
        if (e >= 0)
        {
            if (change_fl == 0)
            {
                y += sy;
            }
            else
            {
                x += sx;
            }

            e -= 1;
        }
        if (change_fl == 0)
        {
             x += sx;
        }
        else
        {
            y += sy;
        }

        e += m;
    }
}
\end{lstlisting}


\begin{lstlisting}[caption=Функция алгоритма построения спектра отрезков по Брезенехему (без многопоточности),
    label={no_par}]
void calculate_spek_no_parallel(const request_t &r, const spektr_t &params)
{
    int spektr = 360;

    point_t p1, p2;
    p1.x = WSIZE_X / 2;
    p1.y = WSIZE_Y / 2;

    for (int i = 0; i < spektr; i += 1)
    {
        p2.x = cos(i * PI / 180) * params.d + WSIZE_X / 2;
        p2.y = sin(i * PI / 180) * params.d + WSIZE_Y / 2;

        // rounding for Bresenham float algorithm
        p1.x = round(p1.x);
        p1.y = round(p1.y);
        p2.x = round(p2.x);
        p2.y = round(p2.y);


        calculate_line(r, p1, p2);
    }
}
\end{lstlisting}

\begin{lstlisting}[caption=Функция алгоритма построения спектра отрезков по Брезенехему (с многопоточностью),
	label={par}]
void calculate_part_spek(const request_t &r, const spektr_t &params, int t_ind)
{
    int spektr = 360;
    int sector_in_t = spektr / params.t_count;
    int part_ind = t_ind * sector_in_t;

    if (t_ind + 1 == params.t_count)
    {
        sector_in_t += (spektr - sector_in_t * params.t_count);
    }

    point_t p1, p2;
    p1.x = WSIZE_X / 2;
    p1.y = WSIZE_Y / 2;

    for (int i = part_ind; i < part_ind + sector_in_t; i += 1)
    {

        p2.x = cos(i * PI / 180) * params.d + WSIZE_X / 2;
        p2.y = sin(i * PI / 180) * params.d + WSIZE_Y / 2;

        // rounding for Bresenham float algorithm
        p1.x = round(p1.x);
        p1.y = round(p1.y);
        p2.x = round(p2.x);
        p2.y = round(p2.y);

        calculate_line(r, pt1, pt2);
    }
}


void calculate_spek_parallel(const request_t &r, const spektr_t &params)
{
    std::vector<std::thread> thrs(params.t_count);

    for (int i = 0; i < params.t_count; i++)
    {
        thrs[i] = std::thread(calculate_part_spek, r, params, i);
    }

    for (int i = 0; i < params.t_count; i++)
    {
        thrs[i].join();
    }
}
\end{lstlisting}

\section{Тестирование}

В таблице \ref{test} приведены функциональные тесты для алгоритмов построения спектра отрезков. Все тесты были пройдены успешно.


\begin{table}[h]
	\begin{center}
		\caption{\label{test} Тесты}
		\begin{tabular}{|c|c|c|c|}
			\hline
			№ & Длина отрезка & Кол-во потоков & Ожидаемый результат \\
			\hline
            1 & 0 & 0 & Сообщение об ошибке \\
            \hline
            2 & 50 & -1 & Сообщение об ошибке \\
            \hline
            3 & -100 & 2 & Сообщение об ошибке \\
            \hline
            4 & 200 & 1 &  Спектр с отрезками длины 200 \\
			\hline
			5 & 300 & 8 & Спектр с отрезками длины 300 \\
			\hline
		\end{tabular}
	\end{center}
\end{table}

\section*{Вывод}

В данном разделе были представлены требования к программному обеспечению и средства реализации, реализованы и протестированы алгоритмы построения спектра отрезков по Брезенхему с распараллеливанием и без него.

