\chapter*{Введение}
\addcontentsline{toc}{chapter}{Введение}

\textbf{Целью данной работы} является изучение организации параллельных вычислений на базе алгоритма построения спектра отрезков по Брезенхему.

\textbf{Многопоточность} — способность центрального процессора (CPU) или одного ядра в многоядерном процессоре одновременно выполнять несколько процессов или потоков, соответствующим образом поддерживаемых операционной системой \cite{multithread}.
Этот подход отличается от многопроцессорности, так как многопоточность процессов и потоков совместно использует ресурсы одного или нескольких ядер: вычислительных блоков, кэш-памяти ЦПУ или буфера перевода с преобразованием (TLB).

В тех случаях, когда многопроцессорные системы включают в себя несколько полных блоков обработки, многопоточность направлена на максимизацию использования ресурсов одного ядра, используя параллелизм на уровне потоков, а также на уровне инструкций.
Поскольку эти два метода являются взаимодополняющими, их иногда объединяют в системах с несколькими многопоточными ЦП и в ЦП с несколькими многопоточными ядрами.

Смысл многопоточности — квазимногозадачность на уровне одного исполняемого процесса.
Значит, все потоки процесса помимо общего адресного пространства имеют и общие дескрипторы файлов. Выполняющийся процесс имеет как минимум один (главный) поток.

Многопоточность стала популярной с конца 1990-х годов, так как усилия по дальнейшему использованию параллелизма на уровне инструкций застопорились.

Далее будут приведены достоинства и недостатки многопоточности.

\textbf{Достоинства}:
\begin{itemize}
	\item облегчение программы посредством использования общего адресного пространства;
	\item меньшие затраты на создание потока в сравнении с процессами;
	\item повышение производительности процесса за счёт распараллеливания процессорных вычислений;
	\item если поток часто теряет кэш, другие потоки могут продолжать использовать неиспользованные вычислительные ресурсы.
\end{itemize}

\textbf{Недостатки}:
\begin{itemize}
	\item несколько потоков могут вмешиваться друг в друга при совместном использовании аппаратных ресурсов;
	\item с программной точки зрения аппаратная поддержка многопоточности является достаточно трудоемкой для программного обеспечения;
	\item проблема планирования потоков;
	\item специфика использования. 
\end{itemize}

Каждый поток в процессе --- это задача, которую должен выполнить процессор. Большинство современных процессоров могут выполнять одновременно две задачи на одном ядре, создавая дополнительное виртуальное ядро. 

Эти процессоры называются многоядерными процессорами. Таким образом, четырехъядерный процессор имеет 8 ядер: 4 физических и 4 виртуальных. Каждое ядро может одновременно выполнять только один поток.


\textbf{Для достижения поставленной цели требуется решить следующие задачи}:
\begin{enumerate}[label={\arabic*)}]
	\item изучение основ параллельных вычислений (многопоточности);
	\item изучение алгоритма Брезенхема для построения отрезков;
	\item разработка параллельной и последовательной версии алгоритма построения спектра отрезков по Брезенхему, то есть с использованием многопоточности и без нее;
	\item реализация параллельной и последовательной версий алгоритма построения спектра отрезков по Брезенхему;
    \item сравнительный анализ времени выполнения реализаций последовательной и параллельной при различном количестве потоков версий алгоритма построения спектра отрезков по Брезенхему;
    \item сравнительный анализ времени выполнения реализаций с использованием многопоточности и без при разной длине отрезка в спектре.
\end{enumerate}