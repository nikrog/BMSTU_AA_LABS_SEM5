\chapter{Технологическая часть}
В данном разделе будут представлены требования к программному обеспечению, средства реализации, листинги кода и тесты.

\section{Требования к программному обеспечению}
Вход: действие (0 --- выход из программы, 1 --- последовательная обработка матриц, 2 --- конвейерная обработка матриц, 3 --- сравнение последоватнельной и конвейерной обработки по времени работы, 4 --- вывод информации об этапах обработки матриц), линейный размер квадратных матриц (в случае замеров времени -- начальный и конечный размер) --- целое положительное число, количество матриц (в случае замеров времени -- минимальное и максимальное количество) --- целое положительное число.

Выход: лог этапов обработки матриц, в случае выбора конвейерной обработки выводятся статистические данные --- минимальное, максимальное, среднее, медианное значение времени нахождения заявки в каждой из очередей, во всех очередях, в системе.

\section{Выбор средств реализации}

В качестве языка программирования для реализации данной лабораторной работы был выбран язык программирования С++ \cite{cpp}. Данный язык программирования позволяет замерить время выполнения работы реализации алгоритма, а также организовать конвейерную обработку данных.

Время выполнения реализаций трехэтапного алгоритма обработки матриц было замерено с помощью функции \textit{std::chrono::system\_clock::now(...)} из библиотеки $chrono$ \cite{chrono}.

В качестве среды разработки был выбран CLion \cite{clion}. 

Графики были построены с использованием языка программирования Python \cite{PythonBook}.

\section{Реализация алгоритмов}
В листингах \ref{mult} -- \ref{det} представлена реализация алгоритмов этапов обработки матриц,
в листинге \ref{lin_alg} --- реализация алгоритма последовательной обработки матриц, а в листингах \ref{conv_alg} -- \ref{st3} --- реализация алгоритма конвейерной обработки матриц.

\captionsetup{singlelinecheck = false, justification=raggedright}
\begin{lstlisting}[caption=Реализация алгоритма возведения матрицы в квадрат, 
    label={mult}]
void mult_matrix(matrix_t &matrix)
{
    std::vector<std::vector<double>> tmp_data;
    tmp_data.resize(matrix.size);
    for (size_t i = 0; i < matrix.size; i++)
    {
        tmp_data[i].resize(matrix.size);
    }
    matrix_t res;
    res.size = matrix.size;
    res.data = tmp_data;
    for (size_t i = 0; i < res.size; i++)
    {
        for (size_t j = 0; j < res.size; j++)
        {
            for (size_t k = 0; k < res.size; k++)
            {
                res.data[i][j] += matrix.data[i][k] * matrix.data[k][j];
            }
        }
    }
    for (size_t i = 0; i < res.size; i++){
        for(size_t j = 0; j < res.size; j++){
            matrix.data[i][j] = res.data[i][j];
        }
    }
}
\end{lstlisting}

\begin{lstlisting}[caption=Реализация алгоритма приведения матрицы к верхнетреугольному виду, 
    label={triangulation}]
void triangulate_matr(matrix_t &matrix)
{
    for (size_t i = 0; i < matrix.size - 1; i++)
    {
        size_t max_col_place = get_max_in_column(matrix, i);

        if (max_col_place != i)
        {
            std::swap(matrix.data[i], matrix.data[max_col_place]);
        }

        for (size_t j = i + 1; j < matrix.size; j++)
        {
            double mult = -matrix.data[j][i] / matrix.data[i][i];

            for (size_t k = i; k < matrix.size; k++)
            {
                matrix.data[j][k] += matrix.data[i][k] * mult;
            }
        }
    }
}

int get_max_in_column(matrix_t matrix, int col_place)
{
    double max_elem = matrix.data[col_place][col_place];
    size_t max_place = col_place;

    for (size_t i = col_place + 1; i < matrix.size; i++)
    {
        if (matrix.data[i][col_place] >= max_elem)
        {
            max_elem = matrix.data[i][col_place];
            max_place = i;
        }
    }

    return max_place;
}
\end{lstlisting}

\begin{lstlisting}[caption=Реализация алгоритма нахождения определителя верхнетругольной матрицы, 
    label={det}]
void get_determinate(matrix_t &matrix)
{
    double det = 1;

    for (size_t i = 0; i < matrix.size; i++)
    {
        det *= matrix.data[i][i];
    }

    matrix.det = det;
}
\end{lstlisting}

\begin{lstlisting}[caption=Реализация алгоритма последовательной обработки матриц, 
    label={lin_alg}]
void linear_alg(int count, size_t size)
{
    std::queue<matrix_t> q1;
    std::queue<matrix_t> q2;
    std::queue<matrix_t> q3;

    queues_t queues = {.q1 = q1, .q2 = q2, .q3 = q3};

    for (int i = 0; i < count; i++)
    {
        matrix_t res = generate_matrix(size);

        queues.q1.push(res);
    }

    for (int i = 0; i < count; i++)
    {
        matrix_t matrix = queues.q1.front();
        stage1_linear(matrix, i + 1); // Stage 1
        queues.q1.pop();
        queues.q2.push(matrix);

        matrix = queues.q2.front();
        stage2_linear(matrix, i + 1); // Stage 2
        queues.q2.pop();
        queues.q3.push(matrix);

        matrix = queues.q3.front();
        stage3_linear(matrix, i + 1); // Stage 3
        queues.q3.pop();
    }
}
\end{lstlisting}


\begin{lstlisting}[caption=Реализация алгоритма конвейерной обработки матриц,
    label={conv_alg}]
void conveyor_alg(int count, size_t size)
{
    std::queue<matrix_t> q1;
    std::queue<matrix_t> q2;
    std::queue<matrix_t> q3;

    queues_t queues = {.q1 = q1, .q2 = q2, .q3 = q3};

    for (int i = 0; i < count; i++)
    {
        matrix_t res = gen_matrix(size);
        q1.push(res);
    }

    std::thread threads[THREADS];

    threads[0] = std::thread(stage1_conv, std::ref(q1), std::ref(q2), std::ref(q3));
    threads[1] = std::thread(stage2_conv, std::ref(q1), std::ref(q2), std::ref(q3));
    threads[2] = std::thread(stage3_conv, std::ref(q1), std::ref(q2), std::ref(q3));

    for (int i = 0; i < THREADS; i++)
    {
        threads[i].join();
    }
}
\end{lstlisting}

\begin{lstlisting}[caption=Реализация алгоритма конвейерной обработки матриц (1-ый поток),
	label={st1}]
void stage1_conv(std::queue<matrix_t> &q1, std::queue<matrix_t> &q2)
{
    int task_num = 1;
    std::mutex m;

    while(!q1.empty())
    {
        m.lock();
        matrix_t matrix = q1.front();
        m.unlock();

        log_conveyor(matrix, task_num++, 1, mult_matrix);
        m.lock();
        q2.push(matrix);
        q1.pop();
        m.unlock();
    }
}
\end{lstlisting}

\begin{lstlisting}[caption=Реализация алгоритма конвейерной обработки матриц (2-ой поток),
	label={st2}]
void stage2_conv(std::queue<matrix_t> &q1, std::queue<matrix_t> &q2, std::queue<matrix_t> &q3)
{
    int task_num = 1;
    std::mutex m;

    do
    {
        m.lock();
        bool is_q2empty = q2.empty();
        m.unlock();
        
        if (!is_q2empty)
        {
            m.lock();
            matrix_t matrix = q2.front();
            m.unlock();

            log_conveyor(matrix, task_num++, 2, triangulate_matr);
            m.lock();
            q3.push(matrix);
            q2.pop();
            m.unlock();
        }
    } while (!q1.empty() || !q2.empty());
}
\end{lstlisting}

\begin{lstlisting}[caption=Реализация алгоритма конвейерной обработки матриц (3-ий поток),
	label={st3}]
void stage3_conv(std::queue<matrix_t> &q1, std::queue<matrix_t> &q2, std::queue<matrix_t> &q3)
{
    int task_num = 1;
    std::mutex m;

    do
    {
        m.lock();
        bool is_q3empty = q3.empty();
        m.unlock();

        if (!is_q3empty)
        {
            m.lock();
            matrix_t matrix = q3.front();
            m.unlock();

            log_conveyor(matrix, task_num++, 3, get_determinate);
            m.lock();
            q3.pop();
            m.unlock();
        }
    } while (!q1.empty() || !q2.empty() || !q3.empty());
}
\end{lstlisting}

\section{Тестирование}

В таблице \ref{test} приведены функциональные тесты для алгоритмов последовательной и конвейерной обработки матриц. Все тесты были пройдены успешно.

\begin{table}[h]
	\begin{center}
		\caption{\label{test} Тесты}
		\begin{tabular}{|c|c|c|c|c|}
			\hline
			№ & Действие & Кол-во матриц & Размер матриц & Ожид. результат \\
			\hline
            1 & 1 & 5 & -2 & Некорректный ввод! \\
            \hline
            2 & 1 & -1 & 5 & Некорректный ввод! \\
            \hline
            3 & -1 & 2 & 2 & Неверное действие! \\
            \hline
            4 & 5 & 2 & 2 & Неверное действие! \\
		\hline
		5 & 1 & 10 & 100 & Лог этапов обработки \\
		\hline
            6 & 2 & 10 & 100 & Лог этапов обработки \\
		\hline
            7 & 3 & 20 -- 50 & 2 & Замеры времени \\
		\hline
            8 & 3 & 10 & 10 -- 100 & Замеры времени \\
		\hline
            9 & 4 &  &  & Этапы обработки (вывод)\\
		\hline
            10 & 0 &  &  & Выход\\
		\hline
		\end{tabular}
	\end{center}
\end{table}
\captionsetup{singlelinecheck = false, justification=centering}

\section*{Вывод}
В данном разделе были представлены требования к программному обеспечению и средства реализации, реализованы и протестированы алгоритмы почледовательной и конвейерной обработки матриц.

