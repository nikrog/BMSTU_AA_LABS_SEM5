\chapter*{Заключение}
\addcontentsline{toc}{chapter}{Заключение}

В результате выполнения лабораторной работы цель достигнута: получен навык организации асинхронного взаимодействия между потоками на примере моделирования конвейера.

В ходе выполнения данной работы были решены все задачи:
\begin{enumerate}[label={\arabic*)}]
    \item изучены основы конвейерной обработки данных;
    \item изучены алгоритмы возведения матрицы в квадрат, приведения матрицы к верхнетреугольному виду и нахождения определителя матрицы;
	\item разработаны параллельная и последовательная версии конвейера с тремя стадиями обработки;
	\item реализованы параллельная и последовательная версии конвейера с тремя стадиями обработки;
	\item проведен сравнительный анализ времени работы последовательной и параллельной реализаций конвейера при различном количестве задач;	
	\item собрана статистика времени ожидания заявок (максимальное, минимальное, среднее, медианное значение) в каждой из очередей (лент) конвейера по отдельности, во всех очередях и времени обработки в системе.
\end{enumerate}


В результате лабораторной работы можно сделать вывод, что использование конвейрной обработки эффективнее последовательной реализации по времени работы, причем преимущество конвейерного алгоритма обработки матриц над последовательным растет с увеличением количества задач.

