\chapter{Аналитическая часть}
В данном разделе будет представлено описание конвейерной обработки данных и этапов обработки матрицы.

\section{Конвейерная обработка данных}
Конвейеризация (или конвейерная обработка) в общем случае основана на разделении подлежащей исполнению функции на более мелкие части, называемые ступенями, и выделении для каждой из них отдельного блока аппаратуры. Так, обработку любой машинной команды можно разделить на несколько этапов, организовав передачу данных от одного этапа к следующему \cite{conveyor}.

Конвейерную обработку можно использовать для совмещения этапов выполнения разных команд. Производительность при этом возрастает, так как одновременно на различных лентах конвейера выполняются несколько команд.
Конвейеризация увеличивает пропускную способность процессора, однако она не сокращает время выполнения отдельной команды.

\section{Описание этапов обработки}
В качестве примера для организации конвейрной обработки будет обрабатываться матрица.

Задача обработки матрицы будет разбита на 3 этапа (ленты):
\begin{enumerate}[label={\arabic*)}]
	\item возведение матрицы в квадрат;
	\item приведение матрицы к верхнетреугольному виду;
        \item нахождение определителя матрицы.
\end{enumerate}
Определитель матрицы будет находиться методом Гаусса.

Чтобы вычислить определитель по методу Гаусса, исходная матрица приводится к верхнетреугольной форме с помощью элементарных преобразований, определитель исходной матрицы не изменится и будет равен произведению элементов, расположенных на главной диагонали верхнетреугольной матрицы \cite{gauss}.

Матрица называется верхнетреугольной, если $a_{ij} > 0$ для всех $i>j$, где $a_{ij}$ --- элемент матрицы, расположенный в i-ой строке в j-ом столбце.
На рисунке \ref{img:triangle.png} представлена верхнетреугольная матрица.

\img{100mm}{triangle.png}{Верхнетреугольная матрица}

\section*{Вывод}
В данном разделе было представлено описание конвейерной обработки данных и этапов обработки матрицы.