\chapter*{Введение}
\addcontentsline{toc}{chapter}{Введение}
\textbf{Целью данной работы} является получение навыка организации асинхронного взаимодействия между потоками на примере моделирования конвейера.

Конвейры широко применяются программистами для решения трудоемких задач, которые можно разделить на этапы, а также в большинстве современных быстродействующих процессоров.

\textbf{Для достижения поставленной цели требуется решить следующие задачи}:
\begin{enumerate}[label={\arabic*)}]
	\item изучение основ конвейерной обработки данных;
	\item изучить алгоритмы возведения матрицы в квадрат, приведения матрицы к верхнетреугольному виду и нахождения определителя матрицы;
        \item разработка параллельной и последовательной версий конвейера с тремя стадиями обработки;
        \item реализация параллельной и последовательной версий конвейера с тремя стадиями обработки;
        \item сравнительный анализ времени работы последовательной и параллельной реализаций конвейера при различном количестве задач;
        \item собрать статистику времени ожидания заявок (максимальное, минимальное, среднее, медианное значение) в каждой из очередей (лент) конвейера по отдельности, во всех очередях и времени обработки в системе.
\end{enumerate}