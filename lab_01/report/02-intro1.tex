\chapter*{Введение}
\addcontentsline{toc}{chapter}{Введение}
\textbf{Цель лабораторной работы} - изучение метода динамического программирования на материале расстояний Дамерау-Левенштейна и Левенштейна.

\textbf{Расстояние Левенштейна (редакционное расстояние)} — метрика, измеряющая по модулю разность между двумя последовательностями символов. Оно определяется как минимальное количество редакторских операций, необходимых для преобразования одной строки в другую.

Редакторские операции:
\begin{enumerate}[label={\arabic*)}]
    \item вставка одного символа;
    \item удаление одного символа;
    \item замена 1 символа.
\end{enumerate}

Операциям, используемым в преобразовании, можно назначить свои цены (штрафы).

\textbf{Расстояние Дамерау — Левенштейна} является модификацией расстояния Левенштейна, а именно к операциям вставки, удаления и замены символов, определённых в расстоянии Левенштейна добавлена операция транспозиции, то есть перестановки даух соседних символов.

Применение редакционных расстояний:
\begin{itemize}
	\item компьютерная лингвистика (например, автозамена в поисковых запросах);
	\item биоинформатика (например, анализ иммунитета, сравнение генов).
\end{itemize}

Для достижения поставленной цели необходимо решить следующие задачи:
\begin{enumerate}[label={\arabic*)}]
	\item изучение расстояний Левенштейна и Дамерау-Левенштейна;
	\item разработка алгоритмов поиска расстояний Левенштейна и Дамерау-Левенштейна;
	\item реализация одного алгоритма поиска расстояния Левенштейна (нерекурсивный с заполнением матрицы расстояний), трех алгоритмов поиска расстояния Дамерау-Левенштейна (нерекурсивный, рекурсивный без кэширования, рекурсивный с кэшированием);
	\item выполнение оценки затрат алгоритмов по памяти;
	\item выполнение замеров процессорного времени работы реализаций алгоритмов:  поиска расстояния Левенштейна (нерекурсивный), поиска расстояния Дамерау-Левенштейна (нерекурсивный, рекурсивный без кэширования, рекурсивный с кэшированием);
	\item сравнительный анализ нерекурсивных алгоритмов поиска расстояний Левенштейна и Дамерау-Левенштейна, рекурсивных алгоритмов поиска расстояния Дамерау-Левенштейна (с кэшированием и без) по затрачиваемым ресурсам.
\end{enumerate}
