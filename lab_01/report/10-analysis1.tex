\chapter{Аналитическая часть}
В данном разделе будут представлены описания алгоритмов нахождения расстояний Левенштейна и Дамерау-Левенштейна.

\textbf{Расстояния Левенштейна и Дамерау–Левенштейна} – это минимальное количество редакторских операций, необходимых для преобразования одной строки в другую.
Различаются данные расстояния только набором допустимых операций.

Введем следующие обозначения операций (в скобках указана цена операции (штраф)):
\begin{itemize}
	\item D (delete) — удаление одного символа (штраф 1);
	\item I (insert) — вставка одного символа (штраф 1);
	\item R (replace) — замена одного символа (штраф 1);
	\item M (match) - совпадение (штраф 0);
	\item X (exchange) - перестановка соседних символов (штраф 1) - только для расстояния Дамерау-Левенштейна.
\end{itemize}

Задача нахождения расстояний Левенштейна и Дамерау–Левенштейна сводится к поиску последовательности операций, дающих в сумме минимально возможный штраф. Данную задачу можно решить с помощью рекуррентных формул, которые будут рассмотрены далее в текущем разделе.

\section{Расстояние Левенштейна}
 
Пусть дано две строки $S_{1}$ и $S_{2}$ длиной $L_{1}$ и $L_{2}$ соответственно. Тогда расстояние Левенштейна можно найти по рекуррентной формуле (\ref{eq:ref1}):

\begin{equation}
	D(S_1[1...i],S_2[1...j]) = \left\{ \begin{array}{ll}
		$0,  i = 0, j = 0$\\
		$j,  i = 0, j > 0$\\
		$i,  i > 0, j = 0$\\
		min(\\
		D(S_1[1...i],S_2[1...j-1]) + 1,\\
		D(S_1[1...i-1], S_2[1...j]) + 1,\\
		D(S_1[1...i-1], S_2[1...j-1]) + \\
		\left[ 
		\begin{array}{c} 
			$0,  $S_1$[i] == $S_2$[j]$\\
			$1, иначе$
		\end{array}
		\label{eq:ref1}
		\right.\\
		)
	\end{array} \right.
\end{equation}

Первые три формулы в системе (1.1) описывают тривиальные случаи: 
\begin{itemize}
    \item совпадение строк, так как обе строки пусты - M (match); 
    \item вставка j символов в пустую строку $S_{1}$ для создания строки-копии $S_{2}$ длиной j;
    \item удаление всех (i символов) из строки $S_{1}$ для совпадения с пустой строкой $S_{2}$. 
\end{itemize}

В последней формуле из системы (1.1) необходимо выбрать минимум из штрафов:
\begin{itemize}
    \item операция вставки символа (I) в $S_{1}$, 
    \item операция удаления символа (D) из $S_{1}$, 
    \item операция замены (R) или совпадения (M), в зависимости от равенства рассматриваемых на данном этапе символов строк ~\cite{Levenshtein}.
\end{itemize}

\section{Расстояние Дамерау-Левенштейна}
 
Расстояние Дамерау-Левенштейна между строками $S_{1}$ и $S_{2}$ (длиной $L_{1}$ и $L_{2}$ соответственно) рассчитывается по схожей с (\ref{eq:ref1}) рекуррентной формуле, добавится еще один возможный вариант в группу min (\ref{eq:ref2}):

\begin{equation}
	\left[ 
	\begin{array}{c} 
		D(S_1[1...i-2],S_2[1...j-2]) + 1, $ если $ $i, j>1, $a_{i-1}=b_{j}, a_{i}=b_{j-1}\\
		\infty $ , иначе$ 
	\end{array}
	\right.\\
	\label{eq:ref2}
\end{equation}

Формула (\ref{eq:ref2}) означает перестановку двух соседних символов в $S_{1}$, если длины обеих строк $L_{1}, L_{2} > 1$, и соседние рассматриваемые символы в строках $S_{1}$ и $S_{2}$ крест-накрест равны. Иначе, если хотя бы одно из условий не выполняется, данная оперция не учитывается при поиске минимума, что и обозначет \infty $ в формуле $ (\ref{eq:ref2}).

Полученная формула для поиска расстояния Дамерау-Левенштейна (\ref{eq:ref3}):

\begin{equation}
	D(S_1[1...i],S_2[1...j]) = \left\{ \begin{array}{ll}
		$0, i = 0, j = 0$\\
		$j, i = 0, j > 0$\\
		$i, i > 0, j = 0$ 	\label{eq:ref3}\\ 
		min(\\
		D(S_1[1...i],S_2[1...j-1]) + 1,\\
		D(S_1[1...i-1], S_2[1...j]) + 1,\\
		D(S_1[1...i-1], S_2[1...j-1]) + \\
		\left[ 
		\begin{array}{c} 
			$0, $S_1$[i] == $S_2$[j]$\\
			$1, иначе$
		\end{array}\\,\\
		
		
		
			\left[ 
		\begin{array}{c} 
			D(S_1[1...i-2],S_2[1...j-2]) + 1, \\
			$ $$ $i, j>1, $ a_i=b_{j-1}, b_j=a_{i-1};\\
			\infty $ , иначе$ 
		\end{array}
		\right.\\
		)
	\end{array} \right.
\end{equation}

\section*{Вывод}
В данном разделе были рассмотрены основные материалы и формулы, которые потребуются далее при разработке и реализации алгоритмов поиска расстояний Левенштейна и Дамерау-Левенштейна.