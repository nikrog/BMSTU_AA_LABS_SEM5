\chapter*{Заключение}
\addcontentsline{toc}{chapter}{Заключение}

В результате выполнения лабораторной работы цель достигнута, а именно при исследовании алгоритмов поиска расстояний Дамерау-Левенштейна и Левенштейна был изучен и применен метод динамического программирования.

В ходе выполнения данной работы были решены следующие задачи:

\begin{itemize}
    \item изучены алгоритмы нахождения расстояний Левенштейна и Дамерау-Левенштейна;
	\item разработаны алгоритмы поиска расстояний Левенштейна и Дамерау-Левенштейна;
	\item реализованы алгоритмы поиска расстояния Левенштейна с заполнением матрицы, Дамерау-Левенштейна с использованием рекурсии и с помощью рекурсивного заполнения матрицы;
	\item выполнена оценка затрат алгоритмов поиска расстояний Левенштейна (итеративный), Дамерау-Левенштейна (итеративный, рекурсивный с кешем, рекурсивный без кеша) по памяти;
	\item выполнены замеры процессорного времени работы реализаций алгоритмов поиска расстояний Левенштейна (итеративный), Дамерау-Левенштейна (итеративный, рекурсивный с кешем, рекурсивный без кеша);
	\item проведен сравнительный анализ нерекурсивной и рекурсивной реализаций алгоритмов поиска расстояний Левенштейна и Дамерау-Левенштейна по времени и памяти;
\end{itemize}

В результате лабораторной работы можно сделать вывод, что итеративная реализация алгоритмов поиска расстояний Левенштейна и Дамерау-Левенштейна существенно выигрывает по времени с увеличением длины строк, но проигрывает по количеству затрачиваемой памяти рекурсивной реализации алгоритмов с кешированием.
