\chapter{Технологическая часть}

В данном разделе будут представлены требования к программному обеспечению, средства реализации, листинги кода и тесты.

\section{Требования к программному обеспечению}
Вход: две строки (регистрозависимые);

Выход: искомое расстояние, посчитанное с помощью реализованных алгоритмов: для расстояния Левенштейна - итерационный, а для расстояния Дамерау-Левенштейна - итерационный, рекурсивный (с кешированием и без кеширования).

\begin{enumerate}
	\item Если на вход программы подаются 2 пустые строки - это корректный ввод, программа должна завершаться без ошибок.
	\item В результате работы программа должна вывести число - расстояние Левенштейна или Дамерау-Левенштейна в зависимости от алгоритма, матрицу при необходимости.
\end{enumerate}


\section{Выбор средств реализации}

В качестве языка программирования для реализации данной лабораторной работы был выбран язык программирования Python  \cite{PythonBook}. Данный язык ускоряет процесс разработки и удобен в использовании.

Процессорное время реализованных алгоритмов было замерено с помощью функции process\_time() из библиотеки time \cite{process_time_text}.

В качестве среды разработки был выбран PyCharm Professional \cite{pycharm}. Данная среда разработки является кросс-платформенной, предоставляет функциональный отладчик, средства для рефакторинга кода и возможность быстрой установки необходимых библиотек при необходимости.

\section{Листинги кода}

В листингах \ref{leven_iter} - \ref{dlev_rec_cash} представлены реализации алгоритмов поиска расстояний Левенштейна и Дамерау-Левенштейна.

\begin{lstlisting}[caption=Функция итеративного алгоритма поиска расстояния Левенштейна с заполнением матрицы расстояний, 
    label={leven_iter}]
	def lowenstein_dist_non_recursive(s1, s2, flag=False):
        # len of string + 1 empty symbol
        n = len(s1) + 1
        m = len(s2) + 1

        matrix = [[0 for i in range(m)] for j in range(n)]

        # trivial cases
        for j in range(1, m):
            matrix[0][j] = j  # INSERT
        for i in range(1, n):
            matrix[i][0] = i  # DELETE

        # filling the rest part of the matrix
        for i in range(1, n):
            for j in range(1, m):
                insert = matrix[i][j - 1] + 1
                delete = matrix[i - 1][j] + 1
                tmp = 1  # REPLACE
                if s1[i - 1] == s2[j - 1]:
                    tmp = 0  # MATCH
                replace = matrix[i - 1][j - 1] + tmp
                matrix[i][j] = min(insert, delete, replace)

        if flag:
            print('Matrix:')
            for row in matrix:
                print(row)

        return matrix[n - 1][m - 1]  # Result
\end{lstlisting}


\begin{lstlisting}[caption=Функция итеративного алгоритма поиска расстояния Дамерау-Левенштейна с заполнением матрицы расстояний,
    label={dleven_iter}]
def damerau_lowenstein_dist_non_recursive(s1, s2, flag=False):
    # len of string + 1 empty symbol
    n = len(s1) + 1
    m = len(s2) + 1

    matrix = [[0 for i in range(m)] for j in range(n)]

    # trivial cases
    for j in range(1, m):
        matrix[0][j] = j  # INSERT
    for i in range(1, n):
        matrix[i][0] = i  # DELETE

    # filling the rest part of the matrix
    for i in range(1, n):
        for j in range(1, m):
            insert = matrix[i][j - 1] + 1
            delete = matrix[i - 1][j] + 1
            tmp = 1  # REPLACE
            if s1[i - 1] == s2[j - 1]:
                tmp = 0  # MATCH
            replace = matrix[i - 1][j - 1] + tmp
            matrix[i][j] = min(insert, delete, replace)
            # i -> i-1, i-1 -> i-2 because string don't have zero symbol in begin
            if i > 1 and j > 1 and s1[i - 1] == s2[j - 2] and s1[i - 2] == s2[j - 1]:
                exchange = matrix[i - 2][j - 2] + 1  # EXCHANGE
                matrix[i][j] = min(matrix[i][j], exchange)

    if flag:
        print('Matrix:')
        for row in matrix:
            print(row)

    return matrix[n - 1][m - 1]  # Result
\end{lstlisting}

\begin{lstlisting}[caption=Функция рекурсивного алгоритма поиска расстояния Дамерау-Левенштейна без кеширования,
	label={dleven_rec}]
def damerau_lowenstein_dist_recursive(s1, s2):
    # trivial cases (exit)
    if not s1:
        return len(s2)
    elif not s2:
        return len(s1)

    insert = damerau_lowenstein_dist_recursive(s1, s2[:-1]) + 1
    delete = damerau_lowenstein_dist_recursive(s1[:-1], s2) + 1
    replace = damerau_lowenstein_dist_recursive(s1[:-1], s2[:-1]) + int(s1[-1] != s2[-1])

    if len(s1) > 1 and len(s2) > 1 and s1[-1] == s2[-2] and s1[-2] == s2[-1]:
        exchange = damerau_lowenstein_dist_recursive(s1[:-2], s2[:-2]) + 1
        return min(insert, delete, replace, exchange)
    else:
        return min(insert, delete, replace)

\end{lstlisting}

\begin{lstlisting}[caption=Функция рекурсивного алгоритма поиска расстояния Дамерау-Левенштейна с кешированием,
	label={dlev_rec_cash}]
INF = -1

def damerau_lowenstein_dist_recursive_cache_helper(s1, s2, matrix):
    # len of string
    len1 = len(s1)
    len2 = len(s2)

    # trivial cases
    if not len1:
        matrix[len1][len2] = len2
    elif not len2:
        matrix[len1][len2] = len1
    else:
        # insert
        if matrix[len1][len2 - 1] == INF:
            damerau_lowenstein_dist_recursive_cache_helper(s1, s2[:-1], matrix)
        # delete
        if matrix[len1 - 1][len2] == INF:
            damerau_lowenstein_dist_recursive_cache_helper(s1[:-1], s2, matrix)
        # replace
        if matrix[len1 - 1][len2 - 1] == INF:
            damerau_lowenstein_dist_recursive_cache_helper(s1[:-1], s2[:-1], matrix)
        # exchange
        #if matrix[len1 - 2][len2 - 2] == INF:
            #damerau_lowenstein_dist_recursive_cache_helper(s1[:-2], s2[:-2], matrix)

        matrix[len1][len2] = min(matrix[len1][len2 - 1] + 1,
                                 matrix[len1 - 1][len2] + 1,
                                 matrix[len1 - 1][len2 - 1] + int(s1[-1] != s2[-1]))

        if len1 > 1 and len2 > 1 and s1[-1] == s2[-2] and s1[-2] == s2[-1]:
            matrix[len1][len2] = min(matrix[len1][len2],
                                     matrix[len1 - 2][len2 - 2] + 1)

        return
        
def damerau_lowenstein_dist_recursive_cache(s1, s2, flag=False):
    # len of string + 1 empty symbol
    n = len(s1) + 1
    m = len(s2) + 1
    matrix = [[-1 for i in range(m)] for j in range(n)]
    damerau_lowenstein_dist_recursive_cache_helper(s1, s2, matrix)

    if flag:
        print('Matrix:')
        for row in matrix:
            print(row)

    return matrix[n - 1][m - 1]
\end{lstlisting}

\section{Тестирование}

В таблице \ref{test} приведены функциональные тесты для алгоритмов вычисления расстояний Левенштейна и Дамерау — Левенштейна. В таблице приняты следующие обозначения: Расст. Л - результат алгоритма поиска расстояния Левенштейна, Расст. Д-Л - результат алгоритма поиска рассотяния Дамерау-Левенштейна). Все тесты были пройдены успешно.

\begin{table}[h]
	\begin{center}
		\caption{\label{test} Тесты}
		\begin{tabular}{|c|c|c|c|c|}
			\hline
			& & & \multicolumn{2}{c|}{\bfseries Ожидаемый результат}    \\ \cline{3-4}\hline
			№ & Строка 1& Строка 2 & Расст. Л & Расст. Д-Л \\ [0.5ex] 
			\hline
			1 & "пустая строка" & "пустая строка" & 0 & 0\\
			\hline
			2 & "пустая строка" & abc & 3 & 3\\
			\hline
			3 & abc & "пустая строка" & 3 & 3\\
			\hline
			4 & art & art & 0 & 0\\
			\hline
			5 & af & a & 1 & 1\\
			\hline
			6 & f & af & 1 & 1\\
			\hline
			7 & гора & горы & 1 & 1\\
			\hline
			8 & 12345 & 12435 & 2 & 1\\
			\hline
			9 & солнце & солнцестояние & 7 & 7\\
			\hline
			10 & солнцестояние & солнце & 7 & 7\\
			\hline
			11 & кот & скат & 2 & 2\\
			\hline
			12 & нитситту & институт & 4 & 2\\
			\hline
			13 & клсас & класс & 2 & 1\\
			\hline
			14 & аргон & арнон & 1 & 1\\
			\hline
		\end{tabular}
	\end{center}
\end{table}


\section*{Вывод}

В данном разделе были представлены требования к программному обеспечению и средства реализации, реализованы и протестированы алгоритмы поиска расстояний: Левенштейна - итерационный, Дамерау-Левенштейна - итерационный, рекурсивный (с кешированием и без)

