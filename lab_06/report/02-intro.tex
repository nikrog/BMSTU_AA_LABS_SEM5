\chapter*{Введение}
\addcontentsline{toc}{chapter}{Введение}
\textbf{Целью данной работы} является получение навыка параметризации методов на примере решения задачи коммивояжера методом на основе муравьиного алгоритма.

Одной из важных задач является поиск оптимальных маршрутов. Такую задачу можно решать полным перебором, но данное решение является крайне неэффективным по времени (имеют большую трудоемкость) при большом числе вершин (городов) в графе (задача поиска оптимального маршрута представляется в виде графа --- набора вершин и рёбер). Существуют эвристические методы решения данной задачи, они не не гарантируют нахождение глобального оптимума (в данном случае --- кратчайшего маршрута), но они эффективнее по времени (имеют более низкую трудоемкость). 

\textbf{Для достижения поставленной цели требуется решить следующие задачи}:
\begin{enumerate}[label={\arabic*)}]
        \item изучение задачи коммивояжера и основ муравьиного алгоритма;
	\item разработка методов решения задачи коммивояжёра --- метод полного перебора и метод на основе муравьиного алгоритма;
        \item реализация разработанных алгоритмов для решения задачи коммивояжера;
	\item выполнение оценки трудоемкости разработанных алгоритмов;
        \item выполнение параметризации метода на основе муравьиного алгоритма по трём его параметрам;
        \item сравнительный анализ реализации алгоритмов решения задачи коммивояжера по полученным результатам (кратчайшее расстояние).
\end{enumerate}