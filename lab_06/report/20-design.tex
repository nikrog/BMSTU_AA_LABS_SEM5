\chapter{Конструкторская часть}
В данном разделе будут представлены схемы алгоритма полного перебора и муравьиного алгоритма решения задачи коммиявояжера, а также проведена оценка трудоемкости алгоритмов.

\section{Алгоритм полного перебора}
На рисунке \ref{img:full_search.png} приведена схема алгоритма полного перебора, на рисунке \ref{img:next_way.png} --- схема алгоритма генерации маршрутов (для перебора всех возможных вариантов) и на рисунке \ref{img:count_len.png} --- схема алгоритма подсчета длины пути.
\clearpage
\img{240mm}{full_search.png}{Схема алгоритма полного перебора}
\img{240mm}{next_way.png}{Схема алгоритма генерации маршрутов}
\img{240mm}{count_len.png}{Схема алгоритма подсчета длины пути}
\clearpage

\section{Муравьиный алгоритм}
На рисунке \ref{img:ant.png} приведена схема муравьиного алгоритма решения задачи коммивояжера.
\clearpage
\imgw{\textwidth}{ant.png}{Схема алгоритма муравьиного алгоритма решения задачи коммивояжера}
\FloatBarrier

\section{Оценка трудоемкости алгоритмов}
Задача коммивояжера является NP-трудной, и точный переборный алгоритм ее решения имеет сложность равную $O(n!)$, где $n$ --- число городов. Сложность муравьиного алгоритма равна $O(t_{max}*m*n^2))$, то есть она зависит от времени жизни колонии, количества городов и количества муравьев в колонии~\cite{np}. В данной реализации количество муравьев равно количеству городов, и трудоемкость муравьиного алгоритма равна $O(t_{max} * n^3)$.
\section*{Вывод}
В данном разделе были разработаны схемы алгоритма полного перебора и муравьиного алгоритма решения задачи коммивояжера, а также была проведена оценка трудоемкости алгоритмов.