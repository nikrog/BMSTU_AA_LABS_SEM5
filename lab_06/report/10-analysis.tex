\chapter{Аналитическая часть}
В данном разделе будут описаны задача коммивояжера, а также методы ее решения --- метод полного перебора и метод на основе муравьиного алгоритма.

\section{Задача коммивояжера}
\textbf{Коммивояжер} --- бродячий торговец. \textbf{Задача коммивояжёра} --- важная задача транспортной логистики, отрасли, занимающейся планированием транспортных перевозок. Коммивояжёру, чтобы распродать нужные и не очень нужные в хозяйстве товары, следует объехать $n$ пунктов и в конце концов вернуться в исходный пункт. Требуется определить наиболее выгодный маршрут объезда. В качестве меры выгодности маршрута может служить суммарное суммарная стоимость пути, или, в простейшем случае, длина маршрута~\cite{voyage_task}. Таким образом, задача коммивояжера заключается в том, чтобы найти такой порядок посещения вершин графа, при котором путь будет минимален по стоимости (у каждого ребра графа есть стоимость --- расстояние между городами, по варианту --- неориентированный граф, то есть в обе стороны одинаковое расстояние), каждая вершина будет посещена лишь один раз, возврат в начальную вершину учитываться не будет, так как по варианту лабораторной работы маршрут --- незамкнутый. Данная задача является NP-трудной~\cite{np}.

\section{Метод полного перебора для решения задачи коммивояжера}
Cуть алгоритма полного перебора для решения задачи коммивояжера заключается в переборе всех вариантов путей и нахождении кратчайшего из них. Преимущество данного метода --- гарантируется нахождение глобального оптимума (в данном случае -- кратчайший путь), недостаток --- большая трудоемкость $O(n!)$, где $n$ --- число городов~\cite{perebor}.

\section{Метод на основе муравьиного алгоритма для решения задачи коммивояжера}
\textbf{Муравьиный алгоритм} \cite{ant} --- метод решения задачи оптимизации, основаный на моделировании поведения колонии муравьев.

Муравьи действуют, руководствуясь органами чувств. 
Каждый муравей оставляет на своём пути феромоны, чтобы другие могли ориентироваться. 
При большом количестве муравьев наибольшее количество феромона остаётся на наиболее посещаемом пути, посещаемость же может быть связана с длинами рёбер (чем короче ребро, тем привлекательнее оно для муравья).

Суть в том, что отдельно взятый муравей мало что может, поскольку он способен выполнять только максимально простые задачи. Но при большом числе других таких муравьев они могут выступать самостоятельными вычислительными единицами. Муравьи используют непрямой обмен информацией через огружающую среду посредством феромона.

Пусть муравей обладает следующими свойствами:
\begin{enumerate}[label=\arabic*)]
	\item зрение --- способность определить привлекательность ребра по его длине;
	\item обоняние --- способность чуять концентрацию феромона;
        \item память --- способность запомнить пройденный маршрут за текущий день.
\end{enumerate}


Функция, характеризующая привлекательность ребра:

\begin{equation}
	\label{d_func}
	\eta_{ij} = 1 / D_{ij},
\end{equation}
где $D_{ij}$ — расстояние от текущего города (вершины графа) $i$ до заданного города $j$.


Формула вычисления вероятности перехода в заданную точку:

\begin{equation}
	\label{posib}
	p_{k,ij} = \begin{cases}
        0, j \in J_k \\
		\frac{\eta_{ij}^{\alpha}\cdot\tau_{ij}^{\beta}}{\sum_{q\notin J_k} \eta^\alpha_{iq}\cdot\tau^\beta_{iq}}, j \notin J_k
	\end{cases}
\end{equation}
где $a$ --- параметр влияния длины пути (коэффициент жадности), $b$ --- параметр влияния феромона (коэффициент стадности), $\tau_{ij}$ --- количество феромонов на ребре $ij$, $\eta_{ij}$ --- привлекательность ребра $ij$, $J_k$ --- список посещённых за текущий день городов.

После завершения передвижения колонии муравьев (ночью, перед наступлением следующего дня), феромон обновляется по формуле:
\begin{equation}
	\label{update_phero_1}
	\tau_{ij}(t+1) = \tau_{ij}(t)\cdot(1-p) + \Delta \tau_{ij}(t),
\end{equation}
где $p \in (0, 1)$ --- коэффициент испарения.

При этом
\begin{equation}
	\label{update_phero_2}
	\Delta \tau_{ij}(t) = \sum_{k=1}^N \Delta \tau^k_{ij}(t),
\end{equation}
где
\begin{equation}
	\label{update_phero_3}
	\Delta\tau^k_{ij}(t) = \begin{cases}
		Q/L_{k}, \textrm{ребро посещено муравьем $k$ в текущий день $t$,} \\
		0, \textrm{иначе}
	\end{cases}
\end{equation}

Поскольку вероятность перехода в заданную точку \ref{posib} не должна быть равна нулю, необходимо обеспечить неравенство $\tau_{ij} (t)$ нулю посредством введения дополительного минимально возможного значения феромона $\tau_{min}$ и в случае, если $\tau_{ij} (t+1)$ принимает значение, меньшее $\tau_{min}$, откатывать значение феромона до этой величины. 


Путь выбирается по следующей схеме.
\begin{enumerate}
	\item Каждый муравей имеет список запретов --- список уже посещенных городов (вершин графа).
	\item Муравьиное зрение отвечает за эвристическое желание посетить вершину.
	\item Муравьиное обоняние отвечает за ощущение феромона на определенном пути (ребре). При этом количество феромона на пути (ребре) в день $t$ обозначается как $\tau_{i, j} (t)$.
	\item После прохождения определенного ребра муравей откладывает на нем некотрое количество феромона, которое показывает оптимальность сделанного выбора, это количество вычисляется по формуле \eqref{update_phero_3}.
\end{enumerate}

\section*{Вывод}
В данном разделе были описаны задача коммивояжера, а также методы ее решения --- метод полного перебора и метод на основе муравьиного алгоритма.