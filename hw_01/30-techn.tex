\chapter{Технологическая часть}

\section{Выбор языка программирования}
При выполнении домашнего задания использовался язык программирования --- Python \cite{PythonBook}.

\section{Исходный код программы}
В листинге \ref{dleven_iter} представлена реализация поиска расстояния Дамерау-Левенштейна с заполнением матрицы расстояний.

\begin{lstlisting}[caption=Реализация итеративного алгоритма поиска расстояния Дамерау-Левенштейна с заполнением матрицы расстояний,
    label={dleven_iter}]
def damerau_lowenstein_dist_non_recursive(s1, s2, flag=False):
    n = len(s1) + 1                                           # 1
    m = len(s2) + 1                                           # 2

    matrix = [[0 for i in range(m)] for j in range(n)]        # 3

    for j in range(0, m):                                     # 4
        matrix[0][j] = j                                      # 5
    for i in range(0, n):                                     # 6
        matrix[i][0] = i                                      # 7

    for i in range(1, n):                                     # 8
        for j in range(1, m):                                 # 9
            insert = matrix[i][j - 1] + 1                     # 10
            delete = matrix[i - 1][j] + 1                     # 11
            tmp = int(s1[i - 1] == s2[j - 1])                 # 12
            replace = matrix[i - 1][j - 1] + tmp              # 13
            matrix[i][j] = min(insert, delete, replace)       # 14
            if i > 1 and j > 1 and 
            s1[i - 1] == s2[j - 2] and s1[i - 2] == s2[j - 1]:# 15
                exchange = matrix[i - 2][j - 2] + 1           # 16
                matrix[i][j] = min(matrix[i][j], exchange)    # 17

    return matrix[n - 1][m - 1]
\end{lstlisting}

\section{Графовые модели алгоритма}
На рисунке \ref{img:GU} представлен граф управления алгоритма.

\imgw{\textwidth}{GU}{Граф управления алгоритма}

На рисунке \ref{img:IG} представлен информационный граф алгоритма.

\imgw{\textwidth}{IG}{Информационный граф алгоритма}

\clearpage

На рисунке \ref{img:OI} представлена операционная история алгоритма для случая, когда условие в строке 15 листинга \ref{dleven_iter} выполняется, то есть принимает значение "истина". В противном случае (значение "ложь" в условии) во вложенном цикле не будут выполняться строки 16 и 17 листинга \ref{dleven_iter}, не будут задействоваться узлы 16 и 17 в операционной истории (узел 15 станет последним в цикле), в остальном она никак не изменится.

\imgw{\textwidth}{OI}{Операционная история алгоритма}

\clearpage

На рисунке \ref{img:II} представлена информационная история алгоритма для случая, когда условие в строке 15 листинга \ref{dleven_iter} выполняется, то есть принимает значение "истина". В противном случае (значение "ложь" в условии) во вложенном цикле не будут выполняться строки 16 и 17 листинга \ref{dleven_iter}, не будут задействоваться узлы 16 и 17 в информационной истории (узел 15 станет последним в цикле, измененная матрица будет передаваться на следующую итерацию цикла с узла 14 вместо узла 17), в остальном она никак не изменится.

\imgw{\textwidth}{II}{Информационная история алгоритма}