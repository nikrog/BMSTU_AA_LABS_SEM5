\chapter{Технологическая часть}

В данном разделе будут представлены требования к программному обеспечению, средства реализации, листинги кода и тесты.

\section{Требования к программному обеспечению}
Вход: две матрицы A и B, количество строк и стобцов в данных матрицах, причем количество столбцов в матрице A должно быть равно количеству строк в матрице B.;

Выход: матрица C --- результат умножения матриц A и B.

Результат умножения введенных пользователем матриц должен быть выведен для каждого из реализованных алгоритмов умножения матриц (стандартный, Винограда, оптимизированный Винограда).

\section{Выбор средств реализации}

В качестве языка программирования для реализации данной лабораторной работы был выбран язык программирования Python  \cite{PythonBook}. В данном языке программирования есть необходимая для замеров процессорного времени библиотека.

Процессорное время реализованных алгоритмов было замерено с помощью функции process\_time() из библиотеки time \cite{process_time_text}.

В качестве среды разработки был выбран PyCharm Professional \cite{pycharm}. Данная среда разработки является кросс-платформенной, предоставляет функциональный отладчик, средства для рефакторинга кода и возможность установки необходимых библиотек при необходимости.

\section{Реализация алгоритмов}

В листингах \ref{stand_mult} -- \ref{vinograd_opt_mult} представлены реализации различных алгоритмов умножения матриц: стандартного, Винограда, оптимизированного Винограда.

\begin{lstlisting}[caption=Реализация стандартного алгоритма умножения матриц, 
    label={stand_mult}]
def standart_mult_matr(matr1, matr2):
    if len(matr1[0]) != len(matr2):
        print("Incorrect size of matrices!")
        return -1

    n = len(matr1)
    m = len(matr1[0])
    q = len(matr2[0])

    res = [[0] * q for i in range(n)]

    for i in range(n):
        for j in range(q):
            for k in range(m):
                res[i][j] = res[i][j] + matr1[i][k] * matr2[k][j]

    return res
\end{lstlisting}


\begin{lstlisting}[caption=Реализация алгоритма Винограда умножения матриц,
    label={vinograd_mult}]
def vinograd_mult_matr(matr1, matr2):
    if len(matr1[0]) != len(matr2):
        print("Incorrect size of matrices!")
        return -1

    n = len(matr1)
    m = len(matr1[0])
    q = len(matr2[0])

    res = [[0] * q for i in range(n)]
    # vector of rows matrix A
    mulh = [0] * n
    for i in range(n):
        for k in range(m // 2):
            mulh[i] = mulh[i] + matr1[i][2 * k] * matr1[i][2 * k + 1]

    # vector of columns matrix B
    mulv = [0] * q
    for i in range(q):
        for k in range(m // 2):
            mulv[i] = mulv[i] + matr2[2 * k][i] * matr2[2 * k + 1][i]

    # filling matrix C
    for i in range(n):
        for j in range(q):
            res[i][j] = - mulh[i] - mulv[j]
            for k in range(m // 2):
                res[i][j] = res[i][j] + (matr1[i][2 * k] + matr2[2 * k + 1][j]) * \
                            (matr1[i][2 * k + 1] + matr2[2 * k][j])

    if m % 2 == 1:
        for i in range(n):
            for j in range(q):
                res[i][j] = res[i][j] + matr1[i][m - 1] * matr2[m - 1][j]

    return res
\end{lstlisting}

\begin{lstlisting}[caption=Реализация оптимизированного алгоритма Винограда умножения матриц,
	label={vinograd_opt_mult}]
def vinograd_optimized_mult_matrix(matr1, matr2):
    if len(matr1[0]) != len(matr2):
        print("Incorrect size of matrices!")
        return -1

    n = len(matr1)
    m = len(matr1[0])
    q = len(matr2[0])

    res = [[0] * q for i in range(n)]

    # vector of rows matrix A
    mulh = [0] * n
    for i in range(n):
        for k in range(m // 2):
            mulh[i] += matr1[i][k << 1] * matr1[i][(k << 1) + 1]

    # vector of columns matrix B
    mulv = [0] * q
    for i in range(q):
        for k in range(m // 2):
            mulv[i] += matr2[k << 1][i] * matr2[(k << 1) + 1][i]

    flag = m % 2
    
    # filling matrix C
    for i in range(n):
        for j in range(q):
            res[i][j] = - mulh[i] - mulv[j]
            for k in range(m // 2):
                res[i][j] += (matr1[i][k << 1] + matr2[(k << 1) + 1][j]) * \
                            (matr1[i][(k << 1) + 1] + matr2[k << 1][j])
            if flag == 1:
                res[i][j] += matr1[i][m - 1] * matr2[m - 1][j]
    return res

\end{lstlisting}

\section{Тестирование}

В таблице \ref{tabular:test_func} приведены функциональные тесты для реализаций стандартного алгоритма умножения матриц, алгоритма Винограда и оптимизированного алгоритма Винограда. Тесты пройдены успешно.

\begin{table}[h!]
	\caption{\label{tabular:test_func} Тестирование реализаций алгоритмов умножения матриц}
	\begin{center}
		\begin{tabular}{c@{\hspace{7mm}}c@{\hspace{7mm}}c@{\hspace{7mm}}c@{\hspace{7mm}}c@{\hspace{7mm}}c@{\hspace{7mm}}}
			\hline
			Матрица A & Матрица B & Ожидаемый результат \\ 
			\hline
			\vspace{4mm}
			$\begin{pmatrix}
				1 & 1 & 1\\
				2 & 2 & 2\\
				3 & 3 & 3
			\end{pmatrix}$ &
			$\begin{pmatrix}
				1 \\
				1 \\
				1 
			\end{pmatrix}$ &
			$\begin{pmatrix}
				3 \\
				6 \\
				9
			\end{pmatrix}$ \\
			\vspace{2mm}
			\vspace{2mm}
			$\begin{pmatrix}
				1 & 1
			\end{pmatrix}$ &
			$\begin{pmatrix}
				1 \\
				1 
			\end{pmatrix}$ &
			$\begin{pmatrix}
				1 & 1 \\
				1 & 1
			\end{pmatrix}$ \\
			\vspace{2mm}
			\vspace{2mm}
			$\begin{pmatrix}
				1 & 1 \\
				1 & 1
			\end{pmatrix}$ &
			$\begin{pmatrix}
				1 & 1\\
				1 & 1
			\end{pmatrix}$ &
			$\begin{pmatrix}
				2 & 2\\
				2 & 2
			\end{pmatrix}$ \\
			\vspace{2mm}
			\vspace{2mm}
			$\begin{pmatrix}
				7
			\end{pmatrix}$ &
			$\begin{pmatrix}
				7
			\end{pmatrix}$ &
			$\begin{pmatrix}
				49
			\end{pmatrix}$ \\
			\vspace{2mm}
			\vspace{2mm}
			$\begin{pmatrix}
				1 & 2 & 3 & 4\\
				5 & 6 & 7 & 8\\
				9 & 1 & 2 & 3
			\end{pmatrix}$ &
			$\begin{pmatrix}
				6 & 7 & 9\\
				-1 & -3 & 4\\
				3 & 7 & 9 \\
				17 & -5 & 6
			\end{pmatrix}$ &
			$\begin{pmatrix}
				81 & 2 & 68\\
				181 & 26 & 180\\
				110 & 59 & 121 \\
			\end{pmatrix}$\\
			$\begin{pmatrix}
				1 \\
				2
			\end{pmatrix}$ &
			$\begin{pmatrix}
				0 \\
			\end{pmatrix}$ &
			$\begin{pmatrix}
				0 \\
				0
			\end{pmatrix}$ \\
			\vspace{2mm}
			\vspace{2mm}
			$\begin{pmatrix}
				1 & 2 & 3
			\end{pmatrix}$ &
			$\begin{pmatrix}
				4 & 5 \\
				6 & 7
			\end{pmatrix}$ &
			Некорректные размеры матриц!\\
		\end{tabular}
	\end{center}
	
\end{table}

\clearpage

\section*{Вывод}

В данном разделе были представлены требования к программному обеспечению и средства реализации, реализованы и протестированы алгоритмы умножения матриц (стандартный, Винограда, оптимизированный Винограда).