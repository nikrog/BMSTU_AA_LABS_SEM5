\chapter*{Введение}
\addcontentsline{toc}{chapter}{Введение}
\textbf{Цель лабораторной работы} --- изучение способов оптимизации алгоритмов на примере алгоритмов умножения матриц.

Термин «матрица» применяется в различных областях, но основное значение данный термин имеет в математике.

\textbf{Матрица} --- математический объект, записываемый в виде прямоугольной таблицы элементов кольца или поля (например, целых или комплексных чисел), которая представляет собой совокупность строк и столбцов, на пересечении которых находятся её элементы. Количество строк и столбцов матрицы задают размер матрицы \cite{matrix}.

Матрицы часто применяются в математике для компактной записи систем линейных алгебраических или дифференциальных уравнений (количество строк матрицы соответствует числу уравнений, а количество столбцов --- количеству неизвестных).

Таким образом, решение систем линейных уравнений сводится к операциям над матрицами, среди которых встречается умножение.

\textbf{Произведением двух матриц А и В} называется матрица С, элемент которой, находящийся на пересечении i-й строки и j-го столбца, равен сумме произведений элементов i-й строки матрицы А на соответствующие (по порядку) элементы j-го столбца матрицы В \cite{matrix2}.

\textbf{Умножение матриц A и B} --- это операция вычисления этой матрицы C.

Для достижения поставленной цели требуется решить следующие задачи:
\begin{enumerate}[label={\arabic*)}]
    \item изучение двух алгоритмов умножения матриц --- стандартный алгоритм, алгоритм Винограда;
    \item разработка трех алгоритмов умножения матриц --- стандартный, Винограда и Винограда с оптимизациями;
    \item реализация трех алгоритмов умножения матриц;
    \item выполнение замеров процессорного времени работы реализаций алгоритмов умножения матриц;
	\item сравнительный анализ трудоемкости реализаций разработанных алгоритмов умножения матриц на основе теоретических расчетов;
	\item выполнение оценки затрат алгоритмов умножения матриц по памяти и сравнительный анализ;
	\item сравнительный анализ процессорного времени работы реализаций разработанных алгоритмов умножения матриц на основе экспериментальных данных.
\end{enumerate}