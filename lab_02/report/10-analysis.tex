\chapter{Аналитическая часть}
В данном разделе будут представлены описания алгоритмов умножения матриц: стандартного, Винограда.

\section{Стандартный алгоритм умножения матриц}

Пусть даны две прямоугольные матрицы $A[N \times M]$ и $B[M \times Q]$:
$$
A = 
\begin{bmatrix} 
	a_{11} & a_{12} & \cdots & a_{1n} \\
	a_{21} & a_{22} & \cdots & a_{2n} \\ 
	\vdots & \vdots & \ddots & \vdots \\ 
	a_{m1} & a_{m2} & \cdots & a_{mn}
\end{bmatrix}
B =   
\begin{bmatrix} 
	b_{11} & b_{12} & \cdots & b_{1q} \\
	b_{21} & b_{22} & \cdots & b_{2q} \\ 
	\vdots & \vdots & \ddots & \vdots \\ 
	b_{n1} & b_{n2} & \cdots & b_{nq}
\end{bmatrix}
$$
Тогда матрица $C[M \times Q]$ --- произведение матриц A и B:
$$
C = 
\begin{bmatrix} 
	c_{11} & c_{12} & \cdots & c_{1q} \\
	c_{21} & c_{22} & \cdots & c_{2q} \\ 
	\vdots & \vdots & \ddots & \vdots \\ 
	c_{m1} & c_{m2} & \cdots & c_{mq}
\end{bmatrix},
$$
в которой каждый элемент вычисляется по следующей формуле: 
\begin{equation}
	\label{eq:N}
	c_{ij} =
	\sum_{k=1}^{n} a_{ik}b_{kj}, \quad (i=\overline{1,m}; j=\overline{1,q})
\end{equation}

Стандартный алгоритм реализует данную формулу.


\section{Алгоритм Винограда умножения матриц}
Если посмотреть на результат умножения двух матриц, то видно, что каждый элемент в нем представляет собой скалярное произведение соответствующих строки и столбца исходных матриц.
Можно заметить также, что такое умножение допускает предварительную обработку, позволяющую часть работы выполнить заранее.

Рассмотрим два вектора $V = (v_1, v_2, v_3, v_4)$ и $W = (w_1, w_2, w_3, w_4)$.
Их скалярное произведение равно: $V \cdot W = v_1w_1 + v_2w_2 + v_3w_3 + v_4w_4$, что эквивалентно (\ref{for:new}):
\begin{equation}
	\label{for:new}
	V \cdot W = (v_1 + w_2)(v_2 + w_1) + (v_3 + w_4)(v_4 + w_3) - v_1v_2 - v_3v_4 - w_1w_2 - w_3w_4.
\end{equation}

Кажется, что второе выражение задает больше работы, чем первое: вместо четырех умножений мы получаем шесть, а вместо трех сложений --- десять. Однако выражение в правой части последнего равенства допускает предварительную обработку, а именно его части можно вычислить заранее и запомнить для каждой строки первой матрицы и для каждого столбца второй. На практике это означает, что над предварительно обработанными элементами нам придется выполнять лишь первые два умножения и последующие пять сложений, а также дополнительно два сложения \cite{vinograd}.

В конце нужно проверить кратность размерности матриц $M$ двум. Если она не кратна двум, то нужно добавить к каждому элементу результирующей матрицы произведение последних элементов соответствующих строки и столбца.


\section*{Вывод}
В данном разделе были рассмотрены основные идеи, лежащие в основе рассматриваемых алгоритмов умножения матриц --- стандартного алгоритма и алгоритма Винограда.