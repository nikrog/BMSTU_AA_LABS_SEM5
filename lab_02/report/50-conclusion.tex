\chapter*{Заключение}
\addcontentsline{toc}{chapter}{Заключение}

В результате выполнения лабораторной работы цель достигнута: изучены способы оптимизации алгоритмов на примере алгоритмов умножения матриц --- стандартный, Винограда.

В ходе выполнения данной работы были решены все задачи:
\begin{enumerate}[label={\arabic*)}]
    \item изучены два алгоритма умножения матриц --- стандартный алгоритм, алгоритм Винограда;
	\item разработаны три алгоритма умножения матриц --- стандартный, Винограда и Винограда с оптимизациями;
	\item реализованы алгоритмы умножения матриц;
	\item выполнены замеры процессорного времени работы реализаций алгоритмов умножения матриц;
	\item проведен сравнительный анализ трудоемкости реализаций разработанных алгоритмов умножения матриц на основе теоретических расчетов;
	\item выполнена оценка и сравнительный анализ затрат алгоритмов умножения матриц по памяти; 
	\item проведен сравнительный анализ процессорного времени работы реализаций разработанных алгоритмов умножения матриц на основе экспериментальных данных.
\end{enumerate}

В результате лабораторной работы можно сделать вывод, что самой эффективной по времени (на основе полученных экспериментальных данных) является реализация оптимизированного алгоритма Винограда умножения матриц, а самым эффективным по памяти является стандартный алгоритм умножения матриц.